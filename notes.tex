\documentclass{article}
\usepackage[utf8]{inputenc}
\usepackage{amsmath}
\usepackage{amssymb}
\usepackage{framed}

\setlength{\parindent}{0pt}

\title{MATH214 Notes}
\author{Tianzong Cheng}
\date{2023-03-07}

\begin{document}

\maketitle

\section{Review}

\subsection{History}

\subsection{Algebraic structures}

\begin{framed}

    \textbf{Span}

    \begin{framed}
        \textbf{Notation}\\
        $span(S)$
    \end{framed}

    \begin{framed}
        \textbf{Definition}\\
        Let $S$ be a subset of a vector space $V$. The smallest vector subspace containing $S$ is called the vector subspace spanned by $S$.
    \end{framed}

    \begin{framed}
        \textbf{Properties}\\
        Let $n\in \mathbb{N}^{*}$ and $S=\{ s_{1},s_{2},\cdots,s_{n}\}$ a subset of a vector space $V$ with $n$ elements. The subspace spanned by $S$ is the set of linear combinations of the $s_{i}$, $1\leq i\leq n$.
    \end{framed}

\end{framed}

\subsection{Linear maps}

\section{Finite dimensional vector spaces}

\subsection{Bases}

\begin{framed}

    \textbf{Sum of subspaces}

    \begin{framed}
        \textbf{Notation}\\
        $V_{1}+V_{2}$
    \end{framed}

    \begin{framed}
        \textbf{Definition}\\
        $V_{1}+V_{2}=\{v_{1}+v_{2},(v1,v2)\in V_{1}\times V_{2}\}$
        \begin{framed}
        \textbf{Remark}\\
        \textit{Sum of subspaces} is sum of values.
        \end{framed}
    \end{framed}

    \begin{framed}
        \textbf{Properties}\\
        \textit{Sum of subspaces} is the smallest subspace containing both $V_{1}$ and $V_{2}$.
    \end{framed}

\end{framed}

\begin{framed}

    \textbf{Direct sum of subspaces}

    \begin{framed}
        \textbf{Notation}\\
        $V_{1}\oplus V_{2}$
    \end{framed}

    \begin{framed}
        \textbf{Definition}\\
        Let $V_{1}$ and $V_{2}$ be two subspaces of a vector space $V$. Then $V$ is called the direct sum of $V_{1}$ and $V_{2}$, denoted $V_{1} \oplus V_{2}$ if $V = V_{1}+V_{2}$ and $V_{1}\cap V_{2} = \{0\}$.
    \end{framed}

    \begin{framed}
        \textbf{Properties}
        \begin{itemize}
            \item For any $v\in V$, $(v_{1},v_{2})\in V_{1}\times V_{2}$ satisfying that $v_{1}+v_{2}=v$ is unique. Proof by contradiction is obvious.
            \item The decomposition of a vector space into a \textit{direct sum of subspaces} is not unique.
        \end{itemize}
    \end{framed}

\end{framed}

\begin{framed}

    \textbf{Spanning set}

    \begin{framed}
    \textbf{Definition}\\
        Let $V$ be a vector space and $S\subset V$. If any element in $V$ can be written as a linear combination of vectors in $S$, then $S$ is called a \textit{spanning set} for $V$.
    \end{framed}

    \begin{framed}
        \textbf{Properties}
        \begin{itemize}
            \item If $S$ is a \textit{spanning set} for $V$, then any superset $S'\supset S$ is also a \textit{spanning set} for $V$.
            \item Let $S$ be a \textit{spanning set} for $V$. A subset $S'$ of $V$ is a \textit{spanning set} for $V$ if and only if any element of $S$ can be written as a linear combination of elements of $S'$.
        \end{itemize}
    \end{framed}

\end{framed}

\begin{framed}

    \textbf{Linearly independent}
    
    \begin{framed}
    \textbf{Definition}\\
    A subset $\mathcal{I}$ of a vector space $V$ is said to be linearly independent if no vector in $I$ can be expressed as a linear combinations of others.
    \begin{framed}
    \textbf{Remark}\\
    "Others" refers to other vectors in $\mathcal{I}$.
    \end{framed}
    
    \end{framed}
    
    \begin{framed}
        \textbf{Properties}
        \begin{itemize}
            \item Let $V$ be a vector space and $\mathcal{I}=\{v_{1},v_{2},\cdot,v_{n}\}\subset V$. Then $\mathcal{I}$ is linearly independent if and only if the unique solution to $\alpha _{1}v_{1}+\alpha _{2}v_{2}+\cdots +\alpha _{n}v_{n}=0$ is $\alpha _{1}=\alpha _{2}=\cdots =\alpha _{n}=0$. Proof by contradiction is still obvious.
            \item Any subset of a linearly independent set is linearly independent.
            \item A linearly independent set cannot contain the vector $0$.
            \item A set with two parallel vectors is linearly dependent, however the converse if false.
        \end{itemize}
    \end{framed}
    
\end{framed}

\section{Questions}

\begin{framed}
    What does this mean?\\
    Any element of $\mathbb{K}^{2}$ can be written as $(x-y)(1,0) + y(1,1)$. So $\mathbb{K}^{2}=span((1,0))+span((1,1))$.
\end{framed}

\end{document}