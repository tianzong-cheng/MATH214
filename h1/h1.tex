\documentclass{article}
\usepackage[utf8]{inputenc}
\usepackage{amsmath}
\usepackage{amssymb}

\setlength{\parindent}{0pt}

\title{MATH214 Homework 1}
\author{Tianzong Cheng}
\date{February 2023}

\begin{document}

\maketitle

\section{Exercise 1}

\subsection{Question 1}

Let $(S,*)$ and $(S',*')$ and $(S'',*'')$ be three algebraic structures. Suppose $f:S \rightarrow S'$, $g:S' \rightarrow S''$ and $h = f \circ g:S \rightarrow S''$.

\begin{equation*}
    \text{Then we have}\qquad
    \left\{\begin{aligned}
         & f(x*y) = f(x) *' f(y)                   \\
         & g(f(x)*'f(y))=g\circ f(x)*''g\circ f(y)
    \end{aligned}
    \right.
\end{equation*}

Thus we have $g\circ f(x*y) = g\circ f(x) *'' g\circ f(y)$, namely $h(x*y) = h(x)*''h(y)$. So the composition of $f$ and $g$ is a homomorphism.

\subsection{Question 2}

According to the definition of isomorphisim, an isomorphisim is a bijection.

Let $f$ be an isomorphisim from $S$ to $S'$ and $a\in S$, $a'\in S'$. Since $f$ is a bijection, we have $f(a)=a'$ and $f^{-1}(a')=a$. We can easily observe that $f^{-1}$ is also bijective becuase $f$ is the inverse function of $f^{-1}$.

Next we need to show that $f^{-1}$ is a homomorphism. $\forall x',y'\in S'$,

\begin{equation*}
    \begin{aligned}
        f^{-1}(x'*'y') & =f^{-1}(f(x)*'f(y))   \\
                       & =f^{-1}(f(x*y))       \\
                       & =x*y                  \\
                       & =f^{-1}(x')*f^{-1}(y)
    \end{aligned}
\end{equation*}

\subsection{Question 3}

a) homomorphism

b) endomorphism

c) automorphism

d) automorphism

e) homomorphism

\section{Exercise 2}

Set 3, 4, 6 are vector subspaces meeting the requirments.

\section{Exercise 3}

\subsection{Question 1}

$\forall \alpha \in V$ and $\forall k \in \mathbb{R}$, we have $k\cdot \alpha=k\cdot (x,y,z)=(kx,ky,kz)$. It's obvious that $kx-2ky+3kz=k(x-2y+3z)=0$. So $V$ is a subspace of $\mathbb{R}^{3}$.

\subsection{Question 2}

A set is a collection of distinct objects.

A vector space is a mathematical structure that consists of a set of vectors, together with operations of addition and scalar multiplication.

\subsection{Question 3}

Since $B\subset C$, we only need to prove $C\subset B$, namely $\forall c\in C$, $c\in B$.

Since $A+B=A+C$, $\forall a\in A$ and $b\in B$, we have

\begin{equation*}
    \left\{\begin{aligned}
        a+b\in A+C \\
        a+c\in A+B
    \end{aligned}
    \right.
\end{equation*}

$A$ is a subspace of $V$, so $0\in A$. If $a=0$, then $c\in A+B$. Since $c\in C$, $c\in B$.

That is, $C\subset b$, $B=C$.

\subsection{Question 4}

First, show that $A\cap (B+(A\cap C))\subset (A\cap B)+(A\cap C)$.

$\forall x\in A\cap (B+(A\cap C))$, we have

\begin{equation*}
    \left\{\begin{aligned}
        x & \in A             \\
        x & \in (B+(A\cap C))
    \end{aligned}
    \right.
\end{equation*}

\subsection{Question 5}

I'm afraid this is beyond me.

\section{Exercise 4}

I'm afraid this is beyond me.

\end{document}